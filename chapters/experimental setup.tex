DyPyBench consists of 50 open-source Python projects from different application domains list in Table \ref{table:50_installed_projects}.
We download the source code of these projects along with their test suite into a Docker image which contains the neural model (LExecutor) and the static (PyCG) and dynamic (DynaPyt) analysis frameworks.
We use the integrated test suites to execute the projects and collect various statistics and data.
In the below sections, we explain the setup for LExecutor, PyCG and DynaPyt.
We also explain the evaluation metrics we use to measure the effectiveness of DyPyBench.

\section{LExecutor}
We use DyPyBench to generate the training data for the neural model of LExecutor.
LExecutor instruments the files and then generates traces from these files during run-time.
LExecutor provides its own instrumenter, and the accepts mandatory two arguments to perform instrumentation.
The first is a JSON file to store the mapping of iids and the second is files to instrument.
To generate the trace files, we execute the test suite consisting of the instrumented files.
The trace files from the above step are used to generate training data in the form of .pt files for the neural model.
For this task, LExecutor provides a module PrepareData which accepts the trace files and the JSON file created in the instrumentation step.
To generate the .pt file, LExecutor works in two modes of abstraction, fine-grained and coarse-grained.
This can be set in the Hyperparameters.py file.
Fine-grained abstraction distinguishes between different values of a type, such as positive, negative and zero in numbers and empty and non-empty for lists, string etc.
Coarse-grained abstraction on the other-hand maps all the values of a type to a single class such as int, string, list etc. \cite{LExecutor_2023}
The Listing \ref{code:Hyperparameters.py} shows the possible entries for the abstraction mode.
In this work, we generate the data using the two abstraction modes shown on line 1 and line 2 in the Listing \ref{code:Hyperparameters.py}. 
The PrepareData module also generates a validation file in the .pt format.
The trace entries are split into training and validation data based on the split provided in the Hyperparameters.py file as shown in Listing \ref{code:Hyperparameters.py} on line 3.
\begin{lstlisting}[caption=Abstraction Modes in LExecutor,label=code:Hyperparameters.py,language=Python]
    value_abstraction = "fine-grained"
    value_abstraction = "coarse-grained-deterministic"
    perc_train = 0.95
\end{lstlisting}

\section{PyCG}
PyCG generates call graph based on static analysis of code.
The call graphs is created in the JSON format, where the key is caller and the value is a list of called functions.
The Listing \ref{code:pycg.json} provides a sample entry from the PyCG.
\begin{lstlisting}[caption=Key-Value Pair in PyCG,label=code:pycg.json,language=python]
{
    "tests.test_autoconfig.test_autoconfig_env": [
        "decouple.AutoConfig",
        "os.path.join",
        "os.path.dirname",
        "mock.patch.object"
    ],
    "decouple.AutoConfig": [],
    "os.path.dirname": [],
    "os.path.join": [],
    "tests.test_autoconfig.test_autoconfig_ini": [
        "decouple.AutoConfig",
        "os.path.join",
        "os.path.dirname",
        "mock.patch.object"
    ]
}
\end{lstlisting}

As we can see from the Listing \ref{code:pycg.json}, the structure of the key is \textit{folder.file.class.method}.
The folder structure for the file is present from the root of the project.
While the other elements are filled in based on the structure of the caller method.
For example, as shown on line 2 in Listing \ref{code:pycg.json} the caller is a function \textit{test\_autoconfig\_env} in the file \textit{test\_autoconfig} present in the folder \textit{tests}.
The structure of the values or callees is \textit{module.function}.
As seen in the Listing \ref{code:pycg.json} on line 4, the called function is \textit{join} of the module \textit{os.path}.
PyCG accepts 3 arguments, the package contains the source files, the entry point files and the output file save the JSON.
In DyPyBench, we provide the package as source code cloned from Git and the test files as entry point files for each project.
The execution of the PyCG module then provides us with the call graph in the mentioned JSON format.

\section{DynaPyt}
DynaPyt instruments the files and the execution of the instrumented code provides us with the dynamic analysis.
DynaPyt, has its own modules for instrumentation and analysis.
We first use the instrumentation module to instrument the source files.
To run the analysis, we create a entry file to run all the test files using the main function provided by the pytest module.
Additionally, we provide the option of import-mode set to importlib to the pytest.main function.
We add the Call Graph analysis is DynaPyt with the function pre\_call hook, and specify this analysis to the analysis module of DynaPyt.
The analysis generates the output in the JSON format, with the key as caller and the value as a list of called functions.
The JSON is stored in a file at the end of the execution of the analysis as provided in the Listing \ref{code:CallGraphAnalysis}.
The Listing \ref{code:dynapyt.json} shows the sample JSON generated from DynaPyt.
\begin{lstlisting}[caption=Key-Value Pair in DynaPyt,label=code:dynapyt.json,language=python]
{
    ".DyPyBench.temp.project22.decouple": [
        "decouple.Undefined",
        "collections.OrderedDict",
        "decouple.AutoConfig",
        "configparser.RawConfigParser.read_file"
    ],
    ".DyPyBench.temp.project22.tests.test_autoconfig.test_autoconfig_env": [
        "decouple.AutoConfig",
        "posixpath.dirname",
        "posixpath.join",
        "mock.mock._patch_object",
        "<decouple.AutoConfig object at 0x7f9479012f20>"
    ],
    ".DyPyBench.temp.project22.tests.test_autoconfig.test_autoconfig_ini_in_subdir": [
        "decouple.AutoConfig",
        "posixpath.dirname",
        "posixpath.join",
        "posix.chdir",
        "mock.mock._patch_object",
        "<decouple.AutoConfig object at 0x7f947859fa60>"
    ]
}
\end{lstlisting}

As can be seen from the Listing \ref{code:dynapyt.json}, the structure of the key is \textit{folder.file.class.method}.
The folder element represents the file path from the root of DyPyBench.
While the other elements are present based on the structure of the caller method.
For example, as shown on line 8 in Listing \ref{code:dynapyt.json} the caller is a function \textit{test\_autoconfig\_env} in the file \textit{test\_autoconfig} present in the folder \textit{/DyPyBench/temp/project22/tests}.
The structure of the values or callees is \textit{module.function}.
As seen in the Listing \ref{code:pycg.json} on line 4, the called function is \textit{OrderedDict} of the module \textit{collections}.
Similarly, on line 11 the module is \textit{posixpath} and the method is \textit{join}.
The callee on line 13 and 21 represents a case where the called function is \textit{AutoConfig} of the module \textit{decouple} with different objects indicated by the hex number at the end.
These are essentially the same callees as provided on line 9 and 16, however, the dynamic nature of generation of the call graph provides such results.

\section{Evaluation Metrics}
In order to evaluate the benchmark created using our approach, we use the following metrics.
Firstly, we evaluate the dynamic nature of the benchmark.
To do so, we perform some statistical analysis on the integrated test suites of projects in the benchmark.
These include running the test suites to find the running time, number of passed and failed test cases for each project and their combination, where running time refers to executing the entire test suite including failed, passed, skipped, xfailed, xpassed and error cases.
Secondly, the efficiency for generating data for neural models is evaluated.
This is done by comparing the number of data points collected and the range of validation accuracy for the model.
Data points in the case of LExecutor refers to the unique use-value events which are used by the neural model to train and validate.
The accuracy for the model is how correctly the model predicts the missing input values using LExecutor for the validation data.
We train the model using the data points we generate and compare the validation accuracy to the original data points.
Lastly, we evaluate the effectiveness to provide data for comparison of static and dynamic analysis for call graphs.
This is done with the help of some statistical comparison of the call graphs generated using the projects in the benchmark.
These comparisons include the number of matching callers and callees in DynaPyt and PyCG.
The match refers to the matching string values of callers as keys and callees as values in DynaPyt and PyCG. 
Furthermore, the percentage match with respect to DynaPyt and PyCG is also calculated.

Overall, these evaluation metrics help us to understand the dynamic nature of the benchmark and its usefulness in usage for generating data to answer some research questions on code analysis.