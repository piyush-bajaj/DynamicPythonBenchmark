Python is a popular interpreted, high-level and general-purpose programming language that is easy to read and has a concise syntax, making it suitable for novice and experienced programmers. It supports various programming paradigms and uses dynamic typing and efficient memory management techniques. Python is also highly extensible through the use of modules, making it a preferred choice for adding programmable interfaces to existing applications \cite{Python_language_wiki}. Python provides dynamic features such as dynamic typing, dynamic object creation, dynamic attribute access, dynamic module loading and dynamic function definition.

Python is a popular language for analysis, bug detection, performance checking, and vulnerability testing due to its easy-to-learn syntax, large community, and extensive libraries. Its strong support for data analysis, visualization, and statistical analysis, as well as cross-platform compatibility, make it an excellent choice for developing cross-platform software and tools. Python also has a range of testing and debugging tools, including built-in frameworks like unittest and pytest, and third-party tools like PyCharm and PyDev. These tools help developers detect and fix bugs and improve code performance.

Java, C++ and Python are commonly used high-level programming languages, which are powerful, object-oriented and have dynamic features which make them ideal for complex applications in software development. The dynamic features of Java include automatic memory management, garbage collection, reflections and exceptions, whereas C++ provides the dynamic features such as templates, dynamic memory allocation and runtime dynamic linking. In order to evaluate these dynamic features of a language, a dynamic benchmark can be used, which involves measuring the performance of a system or application while it is running under real-world conditions.  Unlike static benchmarks, which test a system or application under controlled conditions, dynamic benchmarks attempt to simulate the actual usage patterns of users and applications in order to measure performance more accurately. Java and C++, both have a number of such dynamic benchmarks available such as Java Microbenchmark Harness (JMH) \cite{JMH}, The DeCapo Benchmark \cite{DaCapo_2006}, The Computer Language Benchmarks Game \cite{C++_Benchmark1} and Boost.Benchmarks \cite{Boost_Benchmarks}. Although, Python has some dynamic benchmarks available such as PyPerformance \cite{PyPerformance} and Apache Bench \cite{Apache_Benchmark}, these are not general purpose benchmarks like those provided for Java and C++. These python benchmarks measure the performance of the programs for a single dynamic feature in a specific domain under the real-world conditions.

A general-purpose dynamic benchmark of python programs would help us to analyze different dynamic features of the language such as those mentioned above in real world conditions. And at the same time, it would also encapsulated a wide range of application domains which are brought to the table because of the popularity and wide spread usage of python for building complex applications including machine learning, web development, server development and virtual environments to name a few.  Such a benchmark can be used with existing tools such as systrace \cite{systrace} to run and evaluate the diverse python programs, and at the same time allow new tools such as DynaPyt \cite{DynaPyt2022} to perform the required analysis. This general-purpose benchmark can also be used to generate a dataset for machine learning (ML) and deep learning (DL) tasks of software engineering such as bug detection \cite{DeepBugs2018}, code completion \cite{code_completion}, quality analysis \cite{Code_analysis_1, Code_analysis_2}, code refactoring \cite{code_refactoring}, and testing \cite{testing_1, testing_2, testing_3}. Such ML and DL tasks have the potential to improve the productivity, efficiency and quality of code. However, the accuracy and effectiveness of these tasks depend on the quality and quantity of the data used to train the model.

The past decade has seen tremendous progress in the field of artificial intelligence thanks to the resurgence of neural networks through deep learning. This has helped improve the ability for computers to see, hear, and understand the world around them, leading to dramatic advances in the application of AI to many fields of science and other areas of human endeavor \cite{Machine_Learning_decade}. Machine learning, which is a branch of artificial intelligence (AI) and computer science focuses on the use of data and algorithms to imitate the way that humans learn, gradually improving its accuracy. Over the last couple of decades, the technological advances in storage and processing power have enabled some innovative products based on machine learning, such as Netflix’s recommendation engine and self-driving cars \cite{Machine_Learning}.

Tasks such as bug detection \cite{DeepBugs2018}, code completion \cite{code_completion}, quality analysis \cite{Code_analysis_1, Code_analysis_2}, code refactoring \cite{code_refactoring}, and testing \cite{testing_1, testing_2, testing_3} can be performed using machine learning algorithms by analysing software source code and improving the development process. Such program analysis has the potential to improve the productivity, efficiency and quality of code. Machine learning works by training algorithms on data to make predictions or take actions based on that data. There are three main steps in the machine learning process namely, Data collection and preparation, model training, and finally Model evaluation and deployment. Data collection is a critical step as the quality and quantity of the data used to train a model can greatly impact its accuracy and effectiveness. When collecting data for machine learning in software development, it's important to focus on the right kind of data to train the algorithm on. This data should be relevant, diverse, and representative of the problem being solved. For example, if a machine learning model is being trained to detect bugs in code, the data should consist of code snippets with bugs as well as code snippets without bugs.

The program analysis algorithms of machine learning, either use the code snippets which are available via the source code\cite{static_code_analysis} or logs which are generated by the execution of the software \cite{loglens}. In both of these cases we do not get the detailed information related to the run time behaviour of the executed software. Run time behaviour can provide us a different perspective and has the potential to provide deeper insights which can help us in improving the code and the development process. For example, ***

As described above, DynaPyt is a framework which provides us the insights into the run time behaviour of python programs. Since we are already seeing the usefulness of machine learning in the software engineering tasks, we can combine the best of both of these to achieve a program analysis tool which uses machine learning to improve development process of code using the run time behaviour of python programs. At the time of writing this thesis, there are no framework or benchmark tools available which combine them. With this thesis we provide a framework,  which is a benchmark of executable python software which can be used to generate data set for machine learning tasks in software engineering tasks such as code generation, test case generation etc. for researchers and developers. 

The benchmark consists of 50 executable python software from diverse application domains which Python covers being a highly popular general purpose programming language. By using the DynaPyt framework, the data set generated is able to encapsulate the run time behaviour of python programs. In DyPyBench framework, we provide three different program analysis tasks of software engineering using machine learning approach. The first task *** . The second task ***. Finally, the third task ***. In this thesis, we have further used the machine learning algorithm ** to test and evaluate our framework. The results are ***. 
