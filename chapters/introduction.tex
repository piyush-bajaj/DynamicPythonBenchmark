Python is a popular and versatile programming language that is distinguished by its interpreted nature, high-level syntax, and support for a variety of programming paradigms.
It makes use of dynamic typing and efficient memory management techniques, and its modular design makes it easy to extend and integrate programmable interfaces into pre-existing applications \cite{Python_language_wiki}.
Furthermore, Python provides a variety of dynamic features, such as object creation, attribute access, module loading, and function definition, making it appealing to developers of all skill levels. 

Recent studies have demonstrated Python's widespread usage in various applications, including analysis, bug detection, performance assessment, and vulnerability testing \cite{Python_usage_study1, Python_usage_study2}. This can be attributed to its accessible syntax, extensive community support, and comprehensive libraries. Python's robust support for data analysis, visualization, statistical analysis, and cross-platform compatibility renders it an exceptional choice for the development of cross-platform software and tools. Furthermore, Python is equipped with an assortment of testing and debugging tools, ranging from built-in frameworks like unittest and pytest to third-party tools such as PyCharm \cite{PyCharm_2023} and PyDev \cite{PyDev_2023} for debugging and code refactoring. Other such tools include Pynguin for test generation and DynaPyt for dynamic analysis. These tools aid developers in identifying and resolving bugs while optimizing code performance.

In the realm of high-level programming languages, Python, along with Java and C++, are frequently employed for their potency, object-oriented structure, and dynamic features, rendering them suitable for intricate software development applications. The dynamic features of Java encompass automatic memory management, garbage collection, reflections, and exceptions, while C++ offers templates, dynamic memory allocation, and runtime dynamic linking. To assess the dynamic features of these languages, dynamic benchmarks can be employed, which involve gauging the performance of a system or application under real-world conditions. Contrasting with static benchmarks that evaluate systems or applications under controlled circumstances, dynamic benchmarks endeavor to emulate actual usage patterns of users and applications to attain more accurate performance measurements. Java and C++ boast numerous dynamic benchmarks, such as Java Microbenchmark Harness (JMH) \cite{JMH}, The DaCapo Benchmark \cite{DaCapo_2006}, The Computer Language Benchmarks Game \cite{C++_Benchmark1}, and Boost.Benchmarks \cite{Boost_Benchmarks}. Although Python possesses some dynamic benchmarks, such as PyPerformance \cite{PyPerformance} and Apache Bench \cite{Apache_Benchmark}, they are not general-purpose benchmarks like those offered for Java and C++. Python benchmarks evaluate program performance for a single dynamic feature in a specific domain under real-world conditions.

While the existing python benchmarks evaluate single dynamic feature from a specific domain, a general-purpose benchmark would allow us to analyze the different dynamic features in the real world conditions. Such a benchmark would allow the evaluation of program performance from a wide range of application domains such as machine learning, web development, server developement and virtual environments, brought to the table due to the popularity and wide spread usage of python. With the general-purpose python dynamic benchmark one could generate run-time analysis using tools such as systrace \cite{systrace} and DynaPyt \cite{DynaPyt2022}. A general-purpose benchmark can provide training data for machine learning (ML) and deep learning (DL) tasks of software engineering such as bug detection \cite{DeepBugs2018}, code completion \cite{code_completion}, quality analysis \cite{Code_analysis_1, Code_analysis_2}, code refactoring \cite{code_refactoring}, and testing \cite{testing_1, testing_2, testing_3} which improve the productivity, efficiency and quality of code.

In this work, we provide a general-purpose dynamic benchmark of executable python software, named DyPyBench, which consists of 50 python projects including frameworks and libraries. The entire benchmark, encapsulated inside a docker container, can be setup using the docker image containing the projects, alongside the frameworks such as DynaPyt \cite{DynaPyt2022}, LExecutor \cite{LExecutor_2023} and PyCG \cite{PyCG_2021}. All of the python projects are open-sourced and available on github \cite{github} including the analysis frameworks mentioned above. The chosen projects are popular and belong to diverse application domains such as audio (pydub), command-line tools(thefuck), and computer vision(scikit-learn). The benchmark provided in this work provides a single command line interface to run test suites 
 and dynamic analysis using the DynaPyt framework. Using the same command line interface, DypyBench can run LExecutor to generate traces for training the neural model of LExecutor. PyCG, which helps in generating static call graphs can also be run using the command line interface.
 In essence this work aims to achieve the following goals for the dynamic benchmark:
\begin{itemize}
    \item \textbf{Large-scale} The benchmark should comprise tens of real-world open-source projects, allowing users to evaluate their analyses on a wide range of code.
    \item \textbf{Diverse} The benchmark should contain projects from a diverse range of application domains that reflect the state of today's Python ecosystem.
    \item \textbf{Ready-to-run} There should be a single interface to run each project in the benchmark, making it easy for users to set up and execute the entire benchmark.
    \item \textbf{Ready-to-analyze} To enable dynamic analyses, the executions of all projects in the benchmark should be set up to be analyzed with the DynaPyt framework.
    \item \textbf{Compositional} To help users understand the behavior of specific projects or even individual test cases, it should be easy to run subsets of full benchmark.
    \item \textbf{Long-term} The benchmark should be built using commonly used tools and formats, e.g., pip and Docker, to ensure its longevity.
    \item \textbf{**Extensibility} The benchmark should be easily extensible, allowing users to add, remove or update the tools, frameworks and projects.  
\end{itemize}




The past decade has seen tremendous progress in the field of artificial intelligence thanks to the resurgence of neural networks through deep learning. This has helped improve the ability for computers to see, hear, and understand the world around them, leading to dramatic advances in the application of AI to many fields of science and other areas of human endeavor \cite{Machine_Learning_decade}. Machine learning, which is a branch of artificial intelligence (AI) and computer science focuses on the use of data and algorithms to imitate the way that humans learn, gradually improving its accuracy. Over the last couple of decades, the technological advances in storage and processing power have enabled some innovative products based on machine learning, such as Netflix’s recommendation engine and self-driving cars \cite{Machine_Learning}.

Tasks such as bug detection \cite{DeepBugs2018}, code completion \cite{code_completion}, quality analysis \cite{Code_analysis_1, Code_analysis_2}, code refactoring \cite{code_refactoring}, and testing \cite{testing_1, testing_2, testing_3} can be performed using machine learning algorithms by analysing software source code and improving the development process. Such program analysis has the potential to improve the productivity, efficiency and quality of code. Machine learning works by training algorithms on data to make predictions or take actions based on that data. There are three main steps in the machine learning process namely, Data collection and preparation, model training, and finally Model evaluation and deployment. Data collection is a critical step as the quality and quantity of the data used to train a model can greatly impact its accuracy and effectiveness. When collecting data for machine learning in software development, it's important to focus on the right kind of data to train the algorithm on. This data should be relevant, diverse, and representative of the problem being solved. For example, if a machine learning model is being trained to detect bugs in code, the data should consist of code snippets with bugs as well as code snippets without bugs.

The program analysis algorithms of machine learning, either use the code snippets which are available via the source code\cite{static_code_analysis} or logs which are generated by the execution of the software \cite{loglens}. In both of these cases we do not get the detailed information related to the run time behaviour of the executed software. Run time behaviour can provide us a different perspective and has the potential to provide deeper insights which can help us in improving the code and the development process. For example, ***

As described above, DynaPyt is a framework which provides us the insights into the run time behaviour of python programs. Since we are already seeing the usefulness of machine learning in the software engineering tasks, we can combine the best of both of these to achieve a program analysis tool which uses machine learning to improve development process of code using the run time behaviour of python programs. At the time of writing this thesis, there are no framework or benchmark tools available which combine them. With this thesis we provide a framework,  which is a benchmark of executable python software which can be used to generate data set for machine learning tasks in software engineering tasks such as code generation, test case generation etc. for researchers and developers. 

The benchmark consists of 50 executable python software from diverse application domains which Python covers being a highly popular general purpose programming language. By using the DynaPyt framework, the data set generated is able to encapsulate the run time behaviour of python programs. In DyPyBench framework, we provide three different program analysis tasks of software engineering using machine learning approach. The first task *** . The second task ***. Finally, the third task ***. In this thesis, we have further used the machine learning algorithm ** to test and evaluate our framework. The results are ***. 
