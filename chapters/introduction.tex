Python is a popular high-level programming language that is easy to read and has a concise syntax, making it suitable for novice and experienced programmers. It supports various programming paradigms and uses dynamic typing and efficient memory management techniques. Python is also highly extensible through the use of modules, making it a preferred choice for adding programmable interfaces to existing applications \cite{Python_language_wiki}.

DynaPyt is a tool for dynamic analysis of Python code. It aims to provide insights into the behavior of Python programs, such as performance and memory usage, by analyzing the runtime behavior of code. DynaPyt provides several features, such as the ability to collect data on function calls, memory allocation, and object creation, and to visualize the results in a user-friendly interface. The tool is designed to be easy to use and to work with existing Python code, making it accessible to a wide range of users, from researchers to developers. DynaPyt is based on PyPy, a fast and compliant Python interpreter, and is implemented using PyPy's tracing and profiling capabilities. The tool provides several different types of analysis, including time profiling, memory profiling, and function call tracing, making it possible to get a detailed understanding of the behavior of Python programs. This information can be used to identify performance bottlenecks and optimize the program for better performance.  With its ease of use, flexibility, and range of features, DynaPyt is a valuable tool for anyone working with Python code. \cite{DynaPyt2022}

The past decade has seen tremendous progress in the field of artificial intelligence thanks to the resurgence of neural networks through deep learning. This has helped improve the ability for computers to see, hear, and understand the world around them, leading to dramatic advances in the application of AI to many fields of science and other areas of human endeavor \cite{Machine_Learning_decade}. Machine learning, which is a branch of artificial intelligence (AI) and computer science focuses on the use of data and algorithms to imitate the way that humans learn, gradually improving its accuracy. Over the last couple of decades, the technological advances in storage and processing power have enabled some innovative products based on machine learning, such as Netflix’s recommendation engine and self-driving cars \cite{Machine_Learning}.

Tasks such as bug detection \cite{DeepBugs2018}, code completion \cite{code_completion}, quality analysis \cite{Code_analysis_1, Code_analysis_2}, code refactoring \cite{code_refactoring}, and testing \cite{testing_1, testing_2, testing_3} can be performed using machine learning algorithms by analysing software source code and improving the development process. Such program analysis has the potential to improve the productivity, efficiency and quality of code. Machine learning works by training algorithms on data to make predictions or take actions based on that data. There are three main steps in the machine learning process namely, Data collection and preparation, model training, and finally Model evaluation and deployment. Data collection is a critical step as the quality and quantity of the data used to train a model can greatly impact its accuracy and effectiveness. When collecting data for machine learning in software development, it's important to focus on the right kind of data to train the algorithm on. This data should be relevant, diverse, and representative of the problem being solved. For example, if a machine learning model is being trained to detect bugs in code, the data should consist of code snippets with bugs as well as code snippets without bugs.

The program analysis algorithms of machine learning, either use the code snippets which are available via the source code\cite{static_code_analysis} or logs which are generated by the execution of the software \cite{loglens}. In both of these cases we do not get the detailed information related to the run time behaviour of the executed software. Run time behaviour can provide us a different perspective and has the potential to provide deeper insights which can help us in improving the code and the development process. For example, ***

As described above, DynaPyt is a framework which provides us the insights into the run time behaviour of python programs. Since we are already seeing the usefulness of machine learning in the software engineering tasks, we can combine the best of both of these to achieve a program analysis tool which uses machine learning to improve development process of code using the run time behaviour of python programs. At the time of writing this thesis, there are no framework or benchmark tools available which combine them. With this thesis we provide a framework,  which is a benchmark of executable python software which can be used to generate data set for machine learning tasks in software engineering tasks such as code generation, test case generation etc. for researchers and developers. 

The benchmark consists of 50 executable python software from diverse application domains which Python covers being a highly popular general purpose programming language. By using the DynaPyt framework, the data set generated is able to encapsulate the run time behaviour of python programs. In DyPyBench framework, we provide three different program analysis tasks of software engineering using machine learning approach. The first task *** . The second task ***. Finally, the third task ***. In this thesis, we have further used the machine learning algorithm ** to test and evaluate our framework. The results are ***. 
