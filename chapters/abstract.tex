Python has evolved from a language to write small scripts to one of the most popular programming languages.
% It is heavily used in various domains, e.g., machine learning, data analysis, and server-side web applications.
Besides being popular, Python also is a very dynamic language that does not require the programmer to provide type annotations and allows object structures to be modified during execution.
These two properties make Python a prime candidate for dynamic program analyses to detect programming errors, security vulnerabilities, and performance issues.
To test and evaluate dynamic analyses and runtime systems, researchers and practitioners commonly rely on benchmark suites.
% For example, the DaCapo benchmark for Java, the SPEC CPU benchmark for C/C++, and ScalaBench for Scala offer such benchmarks.
Unfortunately, there currently is no comprehensive benchmark of executable Python projects, which hampers the development of dynamic analyses.

In this work, we present DyPyBench, a dynamic benchmark of executable Python software.
The benchmark is large-scale, diverse, versatile, long-term, ready-to-analyze and ready-to-run and is useful for a wide range of applications. 
The 50 projects in DyPyBench provide test suites that help to perform dynamic analyses.
We provide this benchmark as a Docker image which makes it easier to use for researchers and practitioners.
The benchmark also includes three tools for code analysis tasks, namely, LExecutor, DynaPyt and PyCG.

We evaluate the usage of the benchmark with the three tools integrated in DyPyBench.
Our benchmark provides 41,451 successful test cases for dynamic analyses.
The benchmark is used to generate 547,830 data points for neural model training in LExecutor, which provide an accuracy in the range of 71.86\% and 93.67\%.
Furthermore, we use the benchmark to generate static and dynamic call graphs using the tools PyCG and DynaPyt respectively.
The comparison of these call graphs show us a match of 3,147 caller functions and 2938 callees despite the limitations of DynaPyt and PyCG.
